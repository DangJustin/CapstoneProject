\documentclass{article}

\usepackage{booktabs}
\usepackage{tabularx}

\title{Development Plan\\\progname}

\author{\authname}

\date{}

\input{Comments}
%% Common Parts

\newcommand{\progname}{4G06 - Software Engineering} % PUT YOUR PROGRAM NAME HERE
\newcommand{\authname}{Team \#9, Team Name
\\ Justin Dang - dangj15 
\\ Harris Hamid - hamidh1
\\ Fady Marcos - morocof2 
\\ Riswan Ahsan - ahsanm7
\\ Sheikh Afsar - afsars} % AUTHOR NAMES                  

\usepackage{hyperref}
    \hypersetup{colorlinks=true, linkcolor=blue, citecolor=blue, filecolor=blue,
                urlcolor=blue, unicode=false}
    \urlstyle{same}
                                


\begin{document}

\maketitle

\begin{table}[hp]
\caption{Revision History} \label{TblRevisionHistory}
\begin{tabularx}{\textwidth}{llX}
\toprule
\textbf{Date} & \textbf{Developer(s)} & \textbf{Change}\\
\midrule
9/25/23 & All & Revision 0\\
11/1/23 & Justin Dang & Update to POC demo\\
4/4/24 & Justin Dang & Revision 1\\
\bottomrule
\end{tabularx}
\end{table}

% \wss{Put your introductory blurb here.}

This document provides the development plan for the Housemates app. The housemates app will allow for its users to better communicate with their housemates.  Additionally the app will have a cost management and chore management system to allow for splitting of chores/costs amongst housemates.

\section{Team Meeting Plan}

The team will have a weekly progress check (1 hour on length) on either Monday afternoon during tutorial time (2:30 - 4:30 PM) or Friday afternoon (around 4:00 PM). The purpose of these meetings is to discuss what group member has done for the project in the last week. Additional meetings can be scheduled throughout the week if necessary. At the end of each meeting a discussion of what needs to be accomplished before the next meeting will occur. The details of each meeting will be recorded as an issue on GitHub.


\section{Team Communication Plan}

The team will primarily communicate through Microsoft Teams group chat. Discussions regarding specifics of documentation or source code may occur on an issue on GitHub. Group members can also be contacted by their McMaster email if necessary. All group members are expected to be readily available on these communication channels.

\section{Team Member Roles}

These roles are subject to change throughout the course of the development of \progname{}. All team members are expected to contribute to each of the three subsystems of \progname{} (Bill Management, Task Management, Scheduling).

\subsection*{Justin Dang}

Scrum Master
\begin{itemize}
    \item Make sure group is on track to finishing deliverables
    \item Schedule meetings
    \item Create meeting agendas
\end{itemize}

\subsection*{Harris Hamid}

Team Liaison
\begin{itemize}
	\item Handle external communications
    \item Inform group of any new announcements from Instructor
    \item Handle any internal communication issues in the group
\end{itemize}

\subsection*{Fady Marcos}

Database Engineer
\begin{itemize}
    \item Setup database for \progname{}
    \item Design database schema for \progname{}
    \item Troubleshoot issues with the database for \progname{}
\end{itemize}

\subsection*{Rizwan Ahsan}

UI/UX Design Expert
\begin{itemize}
    \item Design the simple and intuitive UI of \progname{}
    \item Ensure that the visual design of \progname{} is appealing to users
    \item Create surveys to get feedback on user experience for \progname{}
\end{itemize}

\subsection*{Sheikh Afsar}

DevOps Engineer
\begin{itemize}
	\item Set up the CI of \progname{}
    \item Set up project structure for \progname{}
    \item If possible implement CD of \progname{}
\end{itemize}


\section{Workflow Plan}

For the duration of the project, our team will adhere to established guidelines to track and manage all of the code and document modifications. We will also leverage GitHub Actions as our CI tool to automatically build, run tests within PRs and any commits, which determine whether it passes or fails. The workflow is as follows:
\begin{itemize}
    \item Open and describe a GitHub issue
    \item Create a new branch to work on the feature/bug
    \item Work on this feature/bug branch
    \item Commit changes to new branch
    \item Create a Pull request for changes
    \item CI pipeline is triggered
    \item Review and Approve PR if no test failures
    \item Merge changes with main branch
    \item Close GitHub issue 
\end{itemize}

\noindent Issue Tags
\begin{itemize}
    \item Bill Management: Issue deals with bill management system of \progname{}
    \item Task Management: Issue deals with task management system of \progname{}
    \item Scheduling: Issue deals with scheduling system of \progname{}
    \item Account: Issue deals with account system of \progname{}
    \item Miscellaneous: Miscellaneous issue that deals with \progname{} that do not fit with the above system tags.
    \item Documentation: Issue deals with documentation of \progname{}
    \item Deliverable: Deliverable for SFWRENG 4G06
    \item Review-Team06: Document review from team 6
    \item Meeting: Team meeting details
\end{itemize}


    

    
% \begin{itemize}
% 	\item How will you be using git, including branches, pull request, etc.?
% 	\item How will you be managing issues, including template issues, issue
% 	classificaiton, etc.?
% \end{itemize}

\section{Proof of Concept Demonstration Plan}
    One of the main risks in this project is in the client-server interactions that are required for the Housemates application. If the client-server interactions fail then the Housemates application will not be able to work correctly. Another risk is that we won't be able to create one of the main features that are outlined in the problem statement document. As such for this POC demo we are going to focus on showcasing a part of the bill splitting feature of the Housemates application. In the POC demo we will demonstrate a client connecting to a server to split an expense with their group of housemates. This will show both that we can successfully have client server interactions and that one of the major features of the application works.
    
    % This problem boils down to a few goals, one of which is communication through text. The main risk would be whether we are able to communicate with all our housemates on time through the app. Receiving notifications or an alert remains outside the scope for the purpose of POC Demo, we will only be verifying whether text is received to all housemates within the app.
    % We will showcase a user-friendly interface, emphasizing ease of use and efficient navigation within the application. Positive user feedback during the demonstration will indicate that we are on the right track in terms of user experience design.
    % We will also simulate different user scenarios to test the application's scalability. Demonstrating that the system can handle a reasonable number of users without significant performance degradation will be essential to convince ourselves that it can scale effectively

% What is the main risk, or risks, for the success of your project?  What will you
% demonstrate during your proof of concept demonstration to convince yourself that
% you will be able to overcome this risk?

\section{Technology}

\begin{itemize}
\item MongoDB: NoSQL database to store user information.
\item Firebase: Backend database for authentication.
\item React: JavaScript library for front-end development.
\item Express: JavaScript framework for back-end development.
\item Node: JavaScript run-time environment for server.
\item JavaScript: Core programming language of the internet.
\item VSCode: Convenient, good quality IDE to be used for front and back end development.
\item GitHub: To be used for version control, CI, and collaboration.
\item ESLint: Static code analysis tool used to find issues in JavaScript code.

\end{itemize}

\section{Coding Standard}

The coding standard we will be following will be similar to Java style coding. This means that all variable and function names will be written in CamelCase and will have neat accurate comments describing functionality. ESLint will be the linter the used to enforce this coding standard.

\section{Project Scheduling}

\begin{center}
\begin{tabular}{ |c|c| } 
\hline
\textbf{Deliverable} & \textbf{Deadline} \\ 
 \hline
 Problem Statement, POC Plan, Development Plan & September 25 \\ 
 Requirements Document Revision 0  & October 4 \\
Hazard Analysis 0 & October 20  \\ 
 V\&V Plan Revision 0 & November 3 \\ 
Proof of Concept Demonstration & November 13--24 \\
Design Document Revision 0 & January 17 \\
Revision 0 Demonstration & February 5--February 16\\
V\&V Report Revision 0 & March 6 \\
Final Demonstration (Revision 1) & March 24 \\
Final Documentation (Revision 1) & April 4 \\
EXPO Demonstration & April 9 \\
 \hline
\end{tabular}
\end{center}

 % \wss{How will the project be scheduled?}

\end{document}
