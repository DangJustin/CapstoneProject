\documentclass{article}

\usepackage{tabularx}
\usepackage{booktabs}

\title{Problem Statement and Goals\\\progname}

\author{\authname}

\date{}

\input{Comments}
%% Common Parts

\newcommand{\progname}{4G06 - Software Engineering} % PUT YOUR PROGRAM NAME HERE
\newcommand{\authname}{Team \#9, Team Name
\\ Justin Dang - dangj15 
\\ Harris Hamid - hamidh1
\\ Fady Marcos - morocof2 
\\ Riswan Ahsan - ahsanm7
\\ Sheikh Afsar - afsars} % AUTHOR NAMES                  

\usepackage{hyperref}
    \hypersetup{colorlinks=true, linkcolor=blue, citecolor=blue, filecolor=blue,
                urlcolor=blue, unicode=false}
    \urlstyle{same}
                                


\begin{document}

\maketitle

\begin{table}[hp]
\caption{Revision History} \label{TblRevisionHistory}
\begin{tabularx}{\textwidth}{llX}
\toprule
\textbf{Date} & \textbf{Developer(s)} & \textbf{Change}\\
\midrule
9/25/23 & All & Revision 0\\
4/4/24 & Justin Dang & Revision 1\\
\bottomrule
\end{tabularx}
\end{table}

\section{Problem Statement}


 %\wss{You should check your problem statement with the
 % \href{https://github.com/smiths/capTemplate/blob/main/docs/Checklists/ProbState-Checklist.pdf}
 %{problem statement checklist}.}
% \wss{You can change the section headings, as long as you include the required information.}

\subsection{Problem}

With the ongoing affordable housing shortage in Canada many people have been forced to find roommates in order to have a place to live in. This is especially common at universities like McMaster where its extremely common to have housemates while in student housing. While having roommates may help ease financial pressures it can lead to lots of stress. Communication issues can be a huge problem. An application that helps with these common stresses in the roommate life would make it more convenient  for the housemates to live together and overall simplify their lives.


\subsection{Inputs and Outputs}

\subsubsection{Inputs}
The main input for Housemates will be a user entering an information for a chore, bill, or event.
\begin{center}
\begin{tabular}{|p{4cm}|p{4.5cm}|p{3.5cm}|}
\hline
\textbf{Category} & \textbf{Description} & \textbf{Format} \\
\hline
Chore Assignments & Details of assigned household tasks and responsibilities. & Task names, frequency, assignees. \\
\hline
Bill Information & Details of shared expenses, such as rent, internet, utilities, groceries, etc. & Bill amounts, due dates, responsible users. \\
\hline
Event Information & Details of an event & name, time\\
\hline
\end{tabular}
\end{center}

\subsubsection{Outputs}
The main output for Housemates will be webpages that display information for user about their current chores, bills and scheduled events.
\begin{center}
\begin{tabular}{|p{4cm}|p{4.5cm}|p{3.5cm}|}
\hline
\textbf{Category} & \textbf{Description} & \textbf{Format} \\
\hline
Report & Information on  chores, bills, or events for a user & HTML Web page \\
\hline
\end{tabular}
\end{center}

\subsection{Stakeholders}

\begin{itemize}
  \item People with housemates: People with housemates are the primary stakeholders of this application. They can use the application to better simplify life with housemates. As such they will have the greatest influence out of the stakeholders on the requirements of the application during the development process.
  \item Landlords / Property Manager / Housing Association: Landlords would be a secondary stakeholder for the application. Landlords might be interested in using an application like this for their tenants so that they will better be able to communicate with them with respects to household tasks that are required. As such, they might play a minor role on determining the requirements of the application during the design process.
  \item Families: Families can also benefit from the app to have a centralized place for all household matters. They can distribute chores evenly and keep track of bills. It promotes talking openly and encourages users to be more responsible about household duties and bills.
\end{itemize}

\subsection{Environment}

The expected environment for this application will be in web-browsers in the form of a web-app. Users would then have the convenience to access the solution/product on any device they want that suits their needs. 
% \wss{Hardware and software}

\section{Goals}

\begin{center}
\begin{tabular}{|p{6cm}|p{6cm}|}
\hline
\textbf{Goals} & \textbf{Importance}\\

\hline
The application will simplify household task management through a task management system. & This allows streamlining the allocation of chores which in return will reduce conflicts and misunderstandings between housemates. \\
\hline
The application will streamline expense sharing through a cost management system. & This makes it possible for roommates to monitor their expenditure and prevent overspending. Additionally, it encourages each person to make a fair financial contribution. \\
\hline
The application will have a scheduling system that will allow for users to schedule quiet hours & This allows users to focus on their work, sleep, or studies without any interruption. \\
\hline
The application will have a calendar to see scheduled events & This allows users to coordinate their schedules and help them in managing their time in a better way. \\
\hline
The application should integrate the task management, expense sharing, and scheduling system together in an intuitive way  & Integrating these systems together would allow users to better make use of these systems. For example, setting a task to remind a roommate of an expense they should remember to pay. \\
\hline The application will have a straightforward and user-friendly interface that is simple to use for all users, regardless of technical ability. & This allows first time users to be interested in our product and a good experience with the overall product. \\
\hline
\end{tabular}
\end{center}

\section{Stretch Goals}

\begin{center}
\begin{tabular}{|p{6cm}|p{6cm}|}
\hline
\textbf{Goals} & \textbf{Importance}\\
\hline
The finished product will have Google account integration &  The application having google account integration will allow for users to be able to see events scheduled on the app in their Google calendar.\\
\hline
\end{tabular}
\end{center}


\end{document}