\documentclass{article}

\usepackage{float}
\restylefloat{table}

\usepackage{booktabs}

\title{Team Contributions: Rev 0\\\progname}

\author{\authname}

\date{\today}

\input{Comments}
%% Common Parts

\newcommand{\progname}{4G06 - Software Engineering} % PUT YOUR PROGRAM NAME HERE
\newcommand{\authname}{Team \#9, Team Name
\\ Justin Dang - dangj15 
\\ Harris Hamid - hamidh1
\\ Fady Marcos - morocof2 
\\ Riswan Ahsan - ahsanm7
\\ Sheikh Afsar - afsars} % AUTHOR NAMES                  

\usepackage{hyperref}
    \hypersetup{colorlinks=true, linkcolor=blue, citecolor=blue, filecolor=blue,
                urlcolor=blue, unicode=false}
    \urlstyle{same}
                                


\begin{document}

\maketitle

\section{Demo Plans}

For revision 0  we plan to show a minimum viable product for \progname. This would involve showcasing the main 3 features of the housemates application: the bill management functionality, the scheduling functionality, and the task management functionality. Specifically, for the bill management function we will show a bill split between a group of users, for the scheduling functionality we will showcase the creation of an event, and for the task management functionality we will showcase the creation of a task for a group.

% \wss{What you will demonstrating}

\section{Meeting Attendance}

% \wss{For each team member how many team meetings have they attended since the
% POC demo.  This number should be determined from the meeting issues in the
% team's repo.  The first entry in the table should be the total number of team
% meetings held by the team.}

\begin{table}[H]
\centering
\begin{tabular}{ll}
\toprule
\textbf{Student} & \textbf{Meetings}\\
\midrule
Total & 2\\
Justin Dang & 2\\
Harris Hamid & 0\\
Fady Marcos & 2\\
Sheikh Afsar & 1\\
Rizwan Ahsan & 1\\
\bottomrule
\end{tabular}
\end{table}

* As Of Jan 30, 2024
% \wss{If needed, an explanation for the counts can be provided here.}

\section{Lecture Attendance}

% \wss{For each team member how many lectures have they attended since the POC
% demo.  This number should be determined from the lecture issues in the team's
% repo.  The first entry in the table should be the total number of lectures since
% the POC demo.}

\begin{table}[H]
\centering
\begin{tabular}{ll}
\toprule
\textbf{Student} & \textbf{Lectures}\\
\midrule
Total & 2\\
Justin Dang & 2\\
Harris Hamid & 0\\
Fady Marcos & 0\\
Sheikh Afsar & 0\\
Rizwan Ahsan & 0\\
\bottomrule
\end{tabular}
\end{table}

* As Of Jan 30, 2024

\section{Commits}

% \wss{For each team member how many commits to the main branch have been made
% since the POC demo.  The total is the total number of commits for the entire
% team since the POC demo.  The percentage is the percentage of the total commits
% made by each team member.}

\begin{table}[H]
\centering
\begin{tabular}{lll}
\toprule
\textbf{Student} & \textbf{Commits} & \textbf{Percent}\\
\midrule
Total & 11 & 100\% \\
Justin Dang & 7 & 63\% \\
Harris Hamid & 2 & 18\% \\
Fady Marcos & 4 & 36\% \\
Rizwan Ahsan & 1 & 9\% \\
Sheikh Afsar & 4 & 36\% \\
\bottomrule
\end{tabular}
\end{table}

% \wss{If needed, an explanation for the counts can be provided here.  For
% instance, if a team member has more commits to unmerged branches, these numbers
% can be provided here.}

* As Of Jan 30, 2024 \newline * Doesn't include development branches (most of rev 0 development)\\

\section{Issue Tracker}

% \wss{For each team member how many issues have they authored (including open and
% closed issues) and how many have they been assigned (only counting closed
% issues).}

\begin{table}[H]
\centering
\begin{tabular}{lll}
\toprule
\textbf{Student} & \textbf{Authored (O+C)} & \textbf{Assigned (C only)}\\
\midrule
Justin Dang & 6 & 4 \\
Harris Hamid & 0 & 4 \\
Fady Marcos & 1 & 4 \\
Rizwan Ahsan & 0 & 4 \\
Sheikh Afsar & 0 & 4 \\
\bottomrule
\end{tabular}
\end{table}


* As Of Jan 30, 2024 \newline * A lot of tasks were tracked in a Google Docs document


% \wss{If needed, an explanation for the counts can be provided here.}

\section{CICD}

CI/CD will be used in \progname{} to automatically run the tests that will verify that the features of \progname{} work correctly. This will be done with GitHub actions' Node.js workflow.


% \wss{Say how CICD is used in your project}

\end{document}